\documentclass[a2,landscape]{sciposter}
\setmargins[1cm]

\usepackage{csvsimple}	% for marker and target positions
\usepackage{readarray}	% for key=value settings file

\usepackage{siunitx}
\sisetup{detect-all,group-digits=integer}

\usepackage{ifthen}

\usepackage{tikz}

% much of the poster setup is read from the file validationSetup.txt.
% read setup
\readarraysepchar{=}
\readdef{../config/validationSetup.txt}{\fileData}
\readarray*\fileData\setupInfo[-,2]
% turn each key into its own command
% NB: this provides \distance, \markerPosFile and \targetPosFile,
% amongst others
\newcounter{keyCount}
\setcounter{keyCount}{0}%
\whiledo{\value{keyCount} < \setupInfoROWS}{%
	\stepcounter{keyCount}%
	\typeout{\arabic{keyCount}: \setupInfo[\arabic{keyCount},1] -> \setupInfo[\arabic{keyCount},2]}
	\expandafter\xdef\csname \setupInfo[\arabic{keyCount},1]\endcsname{%
		\setupInfo[\arabic{keyCount},2]}%
}
\def\markerSideCm{\markerSide\step}
\def\markerFilename#1{all-markers/#1.png}
\def\targetRadCm{.5*\targetDiameter\step}

\makeatletter
\def\convertto#1#2{\strip@pt\dimexpr #2*65536/\number\dimexpr 1#1}
\makeatother

\newlength{\step}
\ifthenelse{\equal{\mode}{deg}}{%
	% convert 1 deg at viewing distance to cm
	% 2*tan(angle/2)*distance
	\pgfmathparse{(2*tan(.5)*\distance)}
	\edef\gridFac{\pgfmathresult}
	\setlength{\step}{\gridFac cm}
	\def\gridAnnot{Grid: \SI{\gridCols}{\degree} x \SI{\gridRows}{\degree}}
}
{ % else
	% cm mode, hopefully. emit warning if not
	\ifthenelse{\equal{\mode}{cm}}{}
	{ % else, mode is not cm
		\GenericError{}{mode is not set to deg or cm, but \mode. Unsupported}
	}
    % all user-provides measures are in cm, so step is simply 1cm
	\setlength{\step}{1cm}
	\def\gridAnnot{Grid: \SI{\gridCols}{\centi\meter} x \SI{\gridRows}{\centi\meter}}
}

% setup gaze targets
\tikzset{
	% https://tex.stackexchange.com/a/290508
	clip even odd rule/.code={\pgfseteorule},
	invclip/.style={
		clip,insert path=
		[clip even odd rule]{
			[reset cm](-\maxdimen,-\maxdimen)rectangle(\maxdimen,\maxdimen)
		}
	}
}
\newcommand{\gazeTarget}[3]
{
	\ifthenelse{\equal{\targetType}{Tobii}}{%
		\ifthenelse{\equal{\useExactTobiiSize}{1}}{%
		% exact size specifications of Tobii calibration target
			\fill[white] (#1-27.25mm,#2-27.25mm) rectangle (#1+27.25mm,#2+27.25mm);
			\fill[#3]    (#1,#2) circle (21.5mm);
			\fill[white] (#1,#2) circle (11.5mm);
			\fill[#3]    (#1,#2) circle (1.5mm);
		}
		{ % else
			% instead, something defined in visual angle that looks good
			\fill[white] (#1-\targetRadCm,#2-\targetRadCm) rectangle (#1+\targetRadCm,#2+\targetRadCm);
			\fill[#3]    (#1,#2) circle (\targetRadCm);
			\fill[white] (#1,#2) circle (.535*\targetRadCm);
			\fill[#3]    (#1,#2) circle (.0698*\targetRadCm);
		}
	}{}
	\ifthenelse{\equal{\targetType}{Thaler}}{%
		\fill[white]    (#1,#2) circle (\targetRadCm);
		\begin{scope}
			% draw circle, clipping away a cross from it
			\begin{pgfinterruptboundingbox}
				\path[invclip]
				(#1-1.1*\targetRadCm,#2-.25*\targetRadCm) --
				(#1+1.1*\targetRadCm,#2-.25*\targetRadCm) --
				(#1+1.1*\targetRadCm,#2+.25*\targetRadCm) --
				(#1-1.1*\targetRadCm,#2+.25*\targetRadCm);
				\path[invclip]
				(#1-.25*\targetRadCm,#2-1.1*\targetRadCm) --
				(#1+.25*\targetRadCm,#2-1.1*\targetRadCm) --
				(#1+.25*\targetRadCm,#2+1.1*\targetRadCm) --
				(#1-.25*\targetRadCm,#2+1.1*\targetRadCm);
			\end{pgfinterruptboundingbox} 
			\fill[#3] (#1,#2) circle (\targetRadCm);
		\end{scope}
		\fill[#3]    (#1,#2) circle (.25*\targetRadCm);
	}{}
}


\begin{document}

\centering
\begin{tikzpicture}[
		draw,color=black!30,yscale=-1
	]
	\if\showGrid1
		\draw[step=\step, line width=3pt] (0cm,0cm) grid (\step*\gridCols,\step*\gridRows);
	\fi
	
\csvreader{../config/\markerPosFile}{ID=\ID, x=\x, y=\y, rotation_angle=\rotang}%
{
	\coordinate (m) at (
	\x*\step,
	\y*\step
	);
	
	\node[inner sep=0pt] (m2) at (m)			{\includegraphics[width=\markerSideCm,angle=\rotang,origin=c]{\markerFilename{\ID}}};
};


\csvreader{../config/\targetPosFile}{ID=\ID, x=\x, y=\y, color=\tColor}%
{
	\gazeTarget{\x*\step}{\y*\step}{\tColor};
};


	
	% annotations
	\if\showAnnotations1
		\newcounter{mycounter}
		\newlength{\mycounterl}
		\if\showGrid1
			\node[left] at (-0\step, 1\step-0.5\step)     { 1 };
			\node[left] at (-0\step, \gridRows\step-0.5\step) { \gridRows };
		
			\node[] at (1\step-0.5\step    ,-0.5\step) { 1 };
			\node[] at (\gridCols\step-0.5\step,-0.5\step) { \gridCols };
		
			\setcounter{mycounter}{1}
			\setlength{\mycounterl}{\value{mycounter}cm}
			\node[right] at (0,-0.5\step-\mycounterl) {
				\gridAnnot
			};
			\stepcounter{mycounter}
			\setlength{\mycounterl}{\value{mycounter}cm}
		
			\node[right] at (0,-0.5\step-\mycounterl) {
				Cell: \SI{\convertto{cm}{\step}}{\centi\meter} x
				      \SI{\convertto{cm}{\step}}{\centi\meter}
			};
			\stepcounter{mycounter}
			\setlength{\mycounterl}{\value{mycounter}cm}
			
			\ifthenelse{\equal{\mode}{deg}}{%
				\node[right] at (0,-0.5\step-\mycounterl) {
					1 cell = \SI{1}{\degree} at \SI{\distance}{\centi\meter}
				};
			\stepcounter{mycounter}
			\setlength{\mycounterl}{\value{mycounter}cm}
			}
			
				\node[right] at (0,-0.5\step-\mycounterl) {
					Markers are \markerSide x\markerSide~cells, which is
					\convertto{cm}{\markerSideCm}x\SI{\convertto{cm}{\markerSideCm}}{\centi\meter}
				};
		\else
			\setcounter{mycounter}{0}
			\setlength{\mycounterl}{\value{mycounter}cm}
			
			\node[right] at (0,-0.5\step-\mycounterl) {
				Markers are 
				\convertto{cm}{\markerSideCm}x\SI{\convertto{cm}{\markerSideCm}}{\centi\meter}
			};
		\fi
	\fi


\end{tikzpicture}

\end{document}
